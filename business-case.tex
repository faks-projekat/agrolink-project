\documentclass[a4paper,12pt]{article}

% Packages
\usepackage[utf8]{inputenc}
\usepackage[T1]{fontenc}
\usepackage{lmodern}
\usepackage[margin=2.5cm]{geometry}
\usepackage{graphicx}
\usepackage{booktabs}
\usepackage{hyperref}
\usepackage{titlesec}
\usepackage{xcolor}
\usepackage{enumitem}
\usepackage{csquotes}
\usepackage{setspace}
\usepackage{tabularx}
\usepackage{float}

% Define colors
\definecolor{primarycolor}{RGB}{39, 74, 93} % Professional blue
\definecolor{secondarycolor}{RGB}{80, 137, 87} % Subtle green

% Title formatting
\titleformat{\section}{\color{primarycolor}\Large\bfseries}{\thesection}{1em}{}
\titleformat{\subsection}{\color{secondarycolor}\large\bfseries}{\thesubsection}{1em}{}

% Hyperref settings
\hypersetup{
    colorlinks=true,
    linkcolor=primarycolor,
    urlcolor=secondarycolor,
    citecolor=secondarycolor
}

% Document information
\title{{\color{primarycolor}\Huge\textbf{AgroLink}\\[0.5cm]
\Large Poslovni slučaj za digitalnu platformu AgroLink}}
\author{Politehnički fakultet\\Univerzitet u Zenici}
\date{Juni 2025}

\begin{document}

\begin{titlepage}
    \maketitle
    \vfill
    \begin{abstract}
        \noindent 
        \textbf{AgroLink} predstavlja inovativnu digitalnu platformu za direktno povezivanje lokalnih 
        poljoprivrednih proizvođača s krajnjim kupcima na području Zapadnog Balkana. 
        Cilj projekta je eliminacija nepotrebnih posrednika u lancu opskrbe, povećanje 
        profitabilnosti za proizvođače i osiguravanje svježih, lokalnih proizvoda po pristupačnijim 
        cijenama za kupce. Ovaj dokument analizira postojeću tržišnu situaciju, identificira ključne 
        potrebe stakeholdera, predstavlja detaljnu arhitekturu predloženog rješenja, procjenjuje 
        komercijalni potencijal i donosi plan implementacije s analizom rizika. Rezultati 
        analize ukazuju na značajnu priliku za digitalizaciju i optimizaciju regionalnog 
        poljoprivrednog tržišta, uz potencijal stvaranja održivog poslovnog modela s pozitivnim 
        društvenim i ekonomskim učinkom.
    \end{abstract}
    \vfill
    \begin{center}
        \textit{Ključne riječi:} digitalna poljoprivreda, e-tržnica, direktna prodaja, 
        poljoprivredni lanac vrijednosti, lokalna proizvodnja, ruralni razvoj
    \end{center}
\end{titlepage}

\onehalfspacing
\tableofcontents
\newpage

% ----------------------------------------------------
\section{Uvod}
\label{sec:uvod}

\subsection{Kontekst i pozadina projekta}
Poljoprivredni sektor u zemljama Zapadnog Balkana suočava se s nekoliko izazova: fragmentirano tržište, nedovoljna digitalizacija, dugi lanci opskrbe i visoke marže posrednika. Dok s jedne strane mali i srednji proizvođači često ne mogu ostvariti adekvatnu cijenu za svoje proizvode, s druge strane urbani potrošači plaćaju visoke cijene za proizvode upitne svježine. 

Prema podacima FAO-a, preko 40\% poljoprivrednih proizvoda u regiji prodaje se kroz najmanje dva posrednička nivoa, što umanjuje zaradu proizvođača za 30-60\%. Istovremeno, istraživanja pokazuju da više od 70\% potrošača u urbanim sredinama preferira lokalno proizvedenu hranu kad im je dostupna, prvenstveno zbog svježine i percepcije zdravije prehrane \cite{FAO2021}.

\subsection{Svrha dokumenta}
Ovaj dokument predstavlja poslovni slučaj za razvoj digitalne platforme AgroLink koja će adresirati navedene probleme direktnim povezivanjem proizvođača i kupaca. Dokument analizira tržište, identificira ključne funkcionalne zahtjeve, procjenjuje poslovne mogućnosti i predlaže plan implementacije.

% ----------------------------------------------------
\section{Strateški kontekst}
\label{sec:strateski-kontekst}

AgroLink projekt doprinosi nekoliko strateških ciljeva:

\begin{enumerate}
    \item \textbf{Regionalni razvoj} - Podrška održivosti malih i srednjih poljoprivrednih proizvođača u ruralnim područjima regije.
    \item \textbf{Digitalna transformacija} - Modernizacija tradicionalnog sektora kroz digitalne alate i metode.
    \item \textbf{Održiva poljoprivreda} - Smanjenje transportnih distanci i poticanje lokalne proizvodnje.
    \item \textbf{Ekonomska efikasnost} - Optimizacija lanca opskrbe eliminacijom nepotrebnih posrednika.
\end{enumerate}

Platforma je u skladu s Agendom za održivi razvoj EU i regionalnim strategijama pametne specijalizacije koje ističu poljoprivredu i digitalnu transformaciju kao prioritetna područja.

% ----------------------------------------------------
\section{Analiza tržišta}
\label{sec:analiza-trzista}

\subsection{Pregled tržišta}

Poljoprivredni sektor predstavlja značajan dio ekonomije zemalja Zapadnog Balkana, čineći između 7-12\% BDP-a različitih zemalja u regiji i zapošljavajući 15-25\% radne snage. Trenutno stanje digitalizacije u poljoprivrednom sektoru je na niskom nivou, s manje od 20\% poljoprivrednika koji koriste bilo kakve digitalne alate u svom poslovanju \cite{WorldBank2022}.

\subsection{Analiza konkurentskih rješenja}

Pregled postojećih rješenja na tržištu pokazuje određeni broj lokalnih inicijativa, ali s ograničenim dometom, funkcionalnostima ili geografskom pokrivenošću. Tabela \ref{table:konkurencija} prikazuje ključne konkurentske platforme u regiji.

\begin{table}[H]
\centering
\caption{Pregled konkurentskih rješenja na tržištu}
\label{table:konkurencija}
\begin{tabularx}{\textwidth}{|l|X|l|X|}
\hline
\textbf{Rješenje} & \textbf{Prednosti} & \textbf{Nedostaci} & \textbf{Geografski fokus} \\
\hline
Plodovi sela & Direktna prodaja bez posrednika; besplatna promocija proizvoda & Još nema online kupovine & Republika Srpska (BiH) \\
\hline
Farmer.ba & Veliki izbor (600+ proizvoda); detaljna baza podataka & Komercijalni troškovi; ograničeno na BiH & Bosna i Hercegovina \\
\hline
OPGburza.com & Jednostavno objedinjavanje ponude više OPG-ova & Minimalna narudžba često visoka & Hrvatska \\
\hline
Mali proizvođači & Velika zajednica (150,000+ članova); edukacije & Potrebna članarina; aktivni samo u Srbiji & Srbija \\
\hline
Meat Your Farmer & Inovativni model pretplate; 100\% domaći proizvodi & Ograničen na mesne proizvode; dugo čekanje & Srbija \\
\hline
Seosko blago & Raznovrstan asortiman tradicionalnih proizvoda & Fokus na gotove (ne svježe) proizvode & Srbija \\
\hline
\end{tabularx}
\end{table}

\subsection{Segmentacija korisnika}

Identificirane su dvije primarne skupine korisnika:

\begin{itemize}
    \item \textbf{Proizvođači}: Mali i srednji poljoprivredni proizvođači, OPG-ovi, poljoprivredne zadruge, koji traže direktan pristup tržištu i pravednije cijene.
    \item \textbf{Kupci}: Pojedinci i domaćinstva u urbanim sredinama koji traže svježe i kvalitetne proizvode, restorani i ugostiteljski objekti fokusirani na lokalnu hranu, te kompanije s programima društvene odgovornosti.
\end{itemize}

% ----------------------------------------------------
\section{Opis AgroLink rješenja}
\label{sec:opis-rjesenja}

\subsection{Vizija i ključni ciljevi}

AgroLink je zamišljen kao sveobuhvatna platforma koja nadilazi jednostavnu e-trgovinu poljoprivrednih proizvoda, već stvara ekosistem koji podržava cijeli lanac vrijednosti od proizvodnje do konzumacije.

\textbf{Ključni ciljevi:}
\begin{itemize}
    \item Povećanje profitabilnosti malih i srednjih proizvođača za najmanje 25\%
    \item Smanjenje cijene svježih proizvoda za krajnje kupce za 10-15\%
    \item Digitalizacija najmanje 30\% ruralnih poljoprivrednih proizvođača u regiji
    \item Smanjenje rasipanja hrane u lancu opskrbe za 20\%
    \item Stvaranje održivog poslovnog modela s pozitivnim društvenim učinkom
\end{itemize}

\subsection{Funkcionalnosti platforme}

AgroLink platforma objedinjuje nekoliko ključnih funkcionalnosti:

\begin{enumerate}
    \item \textbf{Digitalna tržnica} - Centralno mjesto za direktnu prodaju poljoprivrednih proizvoda
    \item \textbf{Sistem naručivanja i isporuke} - Integrirano praćenje narudžbi od proizvođača do kupca
    \item \textbf{Prediktivna analitika} - Predviđanje ponude i potražnje bazirana na sezonalnosti i povijesnim podacima
    \item \textbf{Alati za proizvođače} - Savjeti o uzgoju, predviđanje cijena, analiza tržišta
    \item \textbf{Analitika za općine i institucije} - Podaci o proizvođačima i kulturama za informirano donošenje odluka
    \item \textbf{Sistem ocjenjivanja i recenzija} - Transparentnost kvalitete i poticaj za kontinuirano unapređenje
\end{enumerate}

\subsection{Tehnička specifikacija i arhitektura}

AgroLink će biti razvijen kao cloud-bazirano rješenje s web i mobilnim aplikacijama, koristeći:

\begin{itemize}
    \item \textbf{Frontend}: React.js i React Native za web i mobilne aplikacije
    \item \textbf{Backend}: Node.js s Express.js frameworkom
    \item \textbf{Baza podataka}: MongoDB za fleksibilnost podatkovnog modela
    \item \textbf{Cloud infrastruktura}: AWS za skalabilnost i pouzdanost
    \item \textbf{Analytics}: Kombinacija TensorFlow za prediktivne modele i PowerBI za poslovnu analitiku
    \item \textbf{Integracije}: Open API za povezivanje s logističkim partnerima, sistemima plaćanja i vanjskim izvorima podataka
\end{itemize}

% ----------------------------------------------------
\section{Poslovni model}
\label{sec:poslovni-model}

\subsection{Izvori prihoda}

AgroLink će koristiti kombinaciju izvora prihoda:

\begin{enumerate}
    \item \textbf{Transakcijska provizija} (5-7\% po transakciji) - Primarna metoda monetizacije
    \item \textbf{Premium članarine za proizvođače} (15€/mjesečno) - Za dodatne funkcionalnosti poput marketinga, analitike i prioritetnog rangiranja
    \item \textbf{Institucionalne pretplate} (50-500€/mjesečno) - Za općine, poljoprivredne komore i istraživačke institucije
    \item \textbf{Logističke usluge} (fiksna naknada po dostavi) - Za organizaciju transporta proizvoda
    \item \textbf{Premium oglašavanje} (po prikazivanju/kliku) - Za partnere i eksterne oglašivače
\end{enumerate}

\subsection{Projekcija troškova}

\begin{table}[H]
\centering
\caption{Procijenjeni troškovi razvoja i operacija}
\begin{tabularx}{\textwidth}{|l|X|r|}
\hline
\textbf{Kategorija troška} & \textbf{Opis} & \textbf{Iznos (€)} \\
\hline
Razvoj platforme & Dizajn, razvoj, testiranje i implementacija inicijalnog rješenja & 120.000 \\
\hline
Marketing & Akvizicija korisnika, brendiranje, kampanje & 50.000 \\
\hline
Operativni troškovi (godišnji) & Hostanje, održavanje, podrška korisnicima & 60.000 \\
\hline
Ljudski resursi (godišnji) & Tim za razvoj, podršku i poslovanje (5 zaposlenih) & 150.000 \\
\hline
\end{tabularx}
\end{table}

\subsection{Projekcija prihoda i povrata investicije}

\begin{table}[H]
\centering
\caption{Projekcija prihoda po godinama}
\begin{tabularx}{\textwidth}{|l|r|r|r|r|r|}
\hline
\textbf{Izvor prihoda} & \textbf{Godina 1} & \textbf{Godina 2} & \textbf{Godina 3} & \textbf{Godina 4} & \textbf{Godina 5} \\
\hline
Transakcijske provizije & 40.000€ & 120.000€ & 250.000€ & 400.000€ & 600.000€ \\
\hline
Premium članarine & 18.000€ & 42.000€ & 90.000€ & 150.000€ & 210.000€ \\
\hline
Institucionalne pretplate & 15.000€ & 30.000€ & 60.000€ & 90.000€ & 120.000€ \\
\hline
Logističke usluge & 10.000€ & 30.000€ & 50.000€ & 80.000€ & 120.000€ \\
\hline
Oglašavanje & 5.000€ & 15.000€ & 35.000€ & 60.000€ & 100.000€ \\
\hline
\textbf{Ukupno} & \textbf{88.000€} & \textbf{237.000€} & \textbf{485.000€} & \textbf{780.000€} & \textbf{1.150.000€} \\
\hline
\end{tabularx}
\end{table}

Očekivani povrat investicije (ROI) je 24 mjeseca od lansiranja, s tačkom pokrića (break-even point) krajem druge godine poslovanja. Dugoročna interna stopa povrata (IRR) procjenjuje se na 42\%.

% ----------------------------------------------------
\section{Implementacijski plan}
\label{sec:implementacijski-plan}

\subsection{Faze razvoja}

Implementacija AgroLink platforme odvijat će se u tri glavne faze:

\begin{enumerate}
    \item \textbf{Faza 1: MVP (6 mjeseci)} - Razvoj osnovnog skupa funkcionalnosti 
    \begin{itemize}
        \item Registracija proizvođača i kupaca
        \item Katalog proizvoda i online narudžbe
        \item Osnovna analitika i izvještavanje
        \item Testiranje s ograničenom grupom korisnika u dvije regije (BiH i Srbija)
    \end{itemize}
    
    \item \textbf{Faza 2: Ekspanzija (12 mjeseci)} - Proširenje funkcionalnosti i tržišta
    \begin{itemize}
        \item Integracija s logističkim partnerima
        \item Napredna analitika i prediktivni modeli
        \item Mobilne aplikacije za Android i iOS
        \item Proširenje na Hrvatsku i Crnu Goru
    \end{itemize}
    
    \item \textbf{Faza 3: Napredne funkcionalnosti (18+ mjeseci)} - Kompletan ekosistem
    \begin{itemize}
        \item AI-podržani savjeti za proizvodnju
        \item Integracija s općinskim i institucionalnim partnerima
        \item Regionalne i sezonske marketinške kampanje
        \item Proširenje na ostale zemlje Zapadnog Balkana
    \end{itemize}
\end{enumerate}

\subsection{Rizici i strategije ublažavanja}

\begin{table}[H]
\centering
\caption{Identificirani rizici i strategije ublažavanja}
\begin{tabularx}{\textwidth}{|l|l|X|}
\hline
\textbf{Rizik} & \textbf{Vjerojatnost/Utjecaj} & \textbf{Strategija ublažavanja} \\
\hline
Niska stopa usvajanja kod proizvođača & Visoka/Visok & Program edukacije i podrške; besplatno korištenje u prvoj godini; partnerstva sa zadrugama \\
\hline
Logistički izazovi u ruralnim područjima & Visoka/Srednji & Partnerstvo s lokalnim dostavnim službama; grupiranje narudžbi; click-and-collect opcije \\
\hline
Konkurencija velikih igrača (npr. Amazon) & Srednja/Visok & Fokus na lokalnost i autentičnost; duboko razumijevanje specifičnih lokalnih potreba \\
\hline
Regulatorni izazovi u različitim zemljama & Srednja/Srednji & Pravni savjetnici u svakoj zemlji; modularni pristup usklađen s lokalnim zakonima \\
\hline
Nepovjerenje kupaca prema online kupovini hrane & Visoka/Srednji & Garancija kvalitete i svježine; transparentan sistem ocjenjivanja; probni period \\
\hline
\end{tabularx}
\end{table}

% ----------------------------------------------------
\section{Očekivani utjecaji}
\label{sec:ocekivani-utjecaji}

\subsection{Ekonomski utjecaji}
\begin{itemize}
    \item Povećanje prihoda za male i srednje poljoprivredne proizvođače za 25-40\%
    \item Stvaranje 50+ direktnih i 500+ indirektnih radnih mjesta u ruralnim područjima
    \item Smanjenje gubitaka u lancu opskrbe za 20\% kroz optimizaciju logistike
\end{itemize}

\subsection{Društveni utjecaji}
\begin{itemize}
    \item Poboljšanje vještina digitalne pismenosti kod najmanje 5.000 poljoprivrednika
    \item Povećanje dostupnosti svježih i lokalnih proizvoda u urbanim središtima
    \item Jačanje veza između ruralnih i urbanih zajednica
\end{itemize}

\subsection{Ekološki utjecaji}
\begin{itemize}
    \item Smanjenje transportnih distanci za 40\% kroz lokaliziranu distribuciju
    \item Reduciranje emisije CO₂ za približno 1.500 tona godišnje
    \item Poticanje održivih poljoprivrednih praksi kroz povećanu transparentnost
\end{itemize}

% ----------------------------------------------------
\section{Zaključak i preporuke}
\label{sec:zakljucak}

Analiza tržišta i konkurencije pokazuje značajnu priliku za AgroLink platformu u regiji Zapadnog Balkana. Trenutna fragmentacija tržišta, nepostojanje sveobuhvatnog regionalnog rješenja, i rastući trend potrošačkih preferencija prema lokalnim i svježim proizvodima stvaraju idealne uvjete za implementaciju predloženog rješenja.

Preporučuje se:
\begin{enumerate}
    \item Pokretanje MVP faze s fokusom na BiH i Srbiju kao testna tržišta
    \item Partnerstvo s min. 3 poljoprivredne zadruge u svakoj zemlji za osiguravanje inicijalne ponude
    \item Osiguravanje inicijalnog financiranja od 200.000€ kroz kombinaciju investicija i grant sredstava
    \item Formiranje multidisciplinarnog tima s iskustvom u agrotech i e-commerce sektorima
\end{enumerate}

Uz strateški pristup implementaciji i adekvatno adresiranje identificiranih rizika, AgroLink ima potencijal postati vodeća digitalna platforma za poljoprivredne proizvode u regiji, stvarajući značajnu poslovnu vrijednost uz istovremeno generiranje pozitivnih društvenih i ekoloških utjecaja.

% ----------------------------------------------------
\begin{thebibliography}{99}
\bibitem{FAO2021} Food and Agriculture Organization (2021). \textit{Digital agriculture report: Rural e-commerce development experience in China}. FAO i ITU.
\bibitem{WorldBank2022} World Bank (2022). \textit{Agriculture Digitalization in Eastern Europe and Central Asia}. Washington, D.C.
\bibitem{EU2023} European Commission (2023). \textit{Farm to Fork Strategy}. Brussels.
\end{thebibliography}

\end{document}
